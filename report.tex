\documentclass{report}
\usepackage{longtable}
\usepackage{adjustbox}
\usepackage[intlimits]{amsmath}
\usepackage{mathtools}
\usepackage{nonfloat}
\usepackage{amsmath}
\usepackage{verbatim}
\usepackage{amsthm}
\usepackage{setspace}
\usepackage[paperwidth=16cm,paperheight=24cm]{geometry}
\usepackage[a4,frame,center]{crop}
%\graphicspath{{/home/abhishek/Desktop/"Text Classification"/Report/figures/}}
\graphicspath{{./figures/}}
\usepackage{lipsum}% dummy text
\usepackage[titles]{tocloft}
\usepackage{amssymb}
\setlength\cftbeforechapskip{100pt}
\renewcommand\cftfigafterpnum{\vskip10pt\par}
\renewcommand\cfttabafterpnum{\vskip10pt\par}
\usepackage[superscript,biblabel]{cite}
\usepackage{tocloft,lipsum,pgffor,sectsty}

\setcounter{tocdepth}{4}% Include up to \subsubsection in ToC




% Font changes to ToC content of sectional units
\renewcommand{\cftpartfont}{\normalfont\sffamily\bfseries}% \part font in ToC
\renewcommand{\cftchapfont}{\normalfont\large\itshape}    % \chapter font in ToC
\renewcommand{\cftsecfont}{\normalfont\slshape}           % \section font in ToC
\renewcommand{\cftsubsecfont}{\normalfont\itshape}        % \subsection font in ToC
\renewcommand{\cftsubsubsecfont}{\normalfont\small}       % \subsubsection font in ToC

% Font changes to document content of sectional units
\renewcommand{\partfont}{\normalfont\HUGE\bfseries}
\renewcommand{\chapterfont}{\normalfont\HUGE\bfseries}
\renewcommand{\sectionfont}{\normalfont\Huge\bfseries}
\renewcommand{\subsectionfont}{\normalfont\Huge\bfseries}

\usepackage{url}
\usepackage{enumitem}
\usepackage{chngcntr}
\usepackage[svgnames]{xcolor}
\usepackage{pgfplotstable}
\usepackage{pgfplotstable,booktabs}
\usepackage{csvsimple}
\usepackage{multicol}
\usepackage[pdfpagemode=FullScreen, colorlinks=true]{hyperref} 
\usepackage{hyperref}
\hypersetup{%
  colorlinks = false,
  linkcolor  = black
}
\usepackage{graphicx}
\usepackage{subcaption}
\usepackage{filecontents}
\newcommand{\plogo}{\fbox{$\mathcal{PL}$}}
\usepackage[utf8]{inputenc}
\usepackage[T1]{fontenc}
\usepackage{fouriernc}
\usepackage[utf8]{inputenc}
\usepackage{titlesec}
%\usepackage{etoolbox}
%\makeatletter
%\patchcmd{\ttlh@hang}{\parindent\z@}{\parindent\z@\leavevmode}{}{}
%\patchcmd{\ttlh@hang}{\noindent}{}{}{}
%\makeatother
\usepackage{titletoc}
\usepackage[figurename=Fig - ]{caption}
\captionsetup[table]{skip=20pt}
\captionsetup[figure]{skip=20pt}
\usepackage[tablename=Table - ]{caption}


\usepackage{setspace}

\usepackage{titlesec}
\titleformat{\chapter}[display]
  {\normalfont\Huge\bfseries}{}{0pt}{}

\usepackage{geometry}
 \geometry{
 a4paper,
 tmargin=20mm,
 bmargin=15mm,
 left=20mm,
 right=20mm,
 top=20mm,
 }

\renewcommand\thesection{\arabic{section}}
\titlecontents{chapter}[1.05em]{\bigskip}%
{\contentslabel[\MakeUppercase{\romannumeral\thecontentslabel}]{1em}\enspace\textsc}%numbered\contentslabel
{\hspace*{-1em}\textsc}%numberless
{\hfill\contentspage}%
%
\titlecontents{section}[1.6em]{\bigskip}%
{\thecontentslabel.\enspace}%numbered
{}%numberless
{\titlerule*[1pc]{.}\contentspage}%

\setcounter{tocdepth}{6}

\usepackage{lipsum}

\newcommand\myfigure[1]{%
\medskip\noindent\begin{minipage}{\columnwidth}
\centering%
#1%
%figure,caption, and label go here
\end{minipage}\medskip}




\newenvironment{changemargin}[1]{%
\item[]}{\end{list}}
\setlength{\columnsep}{0cm}


\titlespacing*{\chapter}{0pt}{0pt}{20pt}

\renewcommand{\bibname}{References}

\usepackage{makeidx}
\makeindex

\begin{document} 

\begin{titlepage}
	\centering
	\includegraphics[width=4cm]{logo.png}\\[.5cm]
	{\scshape\LARGE Indian Institute of \\Information Technology, Allahabad \par}
	\vspace{1cm}
	\rule{\textwidth}{2pt}	
	%\vspace{2pt}\vspace{-\baselineskip}
	%\rule{\textwidth}{0.4pt}
	\vspace{0.1\textheight}
		
	\textcolor{Red}{ 
		{\fontsize{35}{42}\selectfont DeliveryMates}\\[0.5\baselineskip]
	}
	
	\vspace{0.185\textheight} 
	
	\rule{0.3\textwidth}{0.4pt} 
	\begin{multicols}{3} 
	\textcolor{Blue}{
		\begin{flushleft} 
		{\large Ayush Agnihotri}\\[5pt] 
		{\large Nidheesh Pandey}\\[5pt]
		{\large Abhishek Pasi}\\[5pt]
		\end{flushleft}
		}
		\columnbreak
		 
	\textcolor{Blue}{
		\begin{flushleft} 
		{\large IIM2015004}\\[5pt] 
		{\large IIM2015501}\\[5pt]
		{\large ICM2015002}\\[5pt]
		\end{flushleft}
		}
		\columnbreak

	\textcolor{Green}{
		\begin{flushright}
		{\Large \textsc{Under supervision}}\\
		{\large of}\\
		{\Large \textsc{\textbf{Dr. Jagpreet Singh}}}
		\end{flushright}
		}
	\end{multicols}
	\vspace{0.065\textheight} 
	
\hfill


	\rule{\textwidth}{0.4pt} % Thin horizontal rule
	
	\vspace{2pt}\vspace{-\baselineskip} % Whitespace between rules
	
	\rule{\textwidth}{2pt} % Thick horizontal rule
	
\end{titlepage}
\pagebreak

%---------------------------------------------------------------------------------------------------

{\chapter*{ \quad \quad \quad \quad \quad \quad  \Huge \scshape \underline {Declaration} }
\vspace{2.0cm}
\begin{spacing}{0.3}
\fontsize{17}{68}\selectfont\linespread{10} {We hereby declare that the work presented in this project report entitled \textbf{``DeliveryMates''},  submitted towards progress of summer project report of Dual Degree(IT) at \textbf{Indian Institute of Information Technology, Allahabad}, is an authenticated record of our original work carried out from \textbf{May 2018} to \textbf{present} under the guidance of \textbf{Dr. Jagpreet Singh}. Due acknowledgements have been made in the text to all other material used. The project was done in full compliance with the requirements and constraints of the prescribed curriculum.}
\end{spacing}
\vspace{5cm}
\Large
\noindent \textbf{Signature Of Supervisor:}\\
\rule[0.5em]{25em}{0.5pt} % This prints a line for the signature
\vspace{1cm}\\
\noindent \textbf{Date:}\\
\rule[0.5em]{25em}{0.5pt}\\ % This prints a line to write the date
\vspace{1cm}\\
\noindent \textbf{Signature Of Students:}\\
\rule[0.5em]{12em}{0.5pt} % This prints a line for the signature
\vspace{1.5cm}\\

}


%----------------------------------------------------------------------------------------
%---------s-------------------------------------------------------------------------------
%	LIST OF CONTENTS/FIGURES/TABLES PAGES
%----------------------------------------------------------------------------------------

\pagenumbering{arabic}
{ \doublespacing
\pagenumbering{Roman}
\tableofcontents % Prints the main table of contents

\addtocontents{toc}{~\hfill\textbf{Page}\par}
\pagebreak
\listoffigures % Prints the list of figures
\pagebreak
\listoftables % Prints the list of tables
\pagebreak

}

%-----------------------------------------------------------------------------------------
\title{\Huge  DeliveryMates\linebreak}
\date{}
\maketitle
\setlength{\columnsep}{0.7cm}
\pagenumbering{arabic}
%\begin{multicols}{1}
\chapter{Abstract}
\par \Large \textit { }

%---------------------------------------------------------------------

\chapter{Introduction and Motivation}

\par \Large The world is getting flooded by textual data everyday. It is very important to extract infromation from this data for better understanding of

%---------------------------------------------------------------------------------------------------

\chapter{Related work}
\par As you already know,  we have made an android application which helps people shop from local shops, it will be helpful to discuss what work has been already done in this space.

\par Let\textquotesingle s have a look at the local shopping market space. While amazon is leader in online shopping and e-commerce, it has also entered into grocery delivering area. Ever heard of Google selling groceries and small daily use items? Yes it is true, Google which is one of the biggest technology company has also entered into this space with its
\textbf{Google express}. From what we have seen it has partnered  with various big store chains like walmart and offers free delivery over a certain cost. Although, it is not available in India as of now.

\par Taxi booking apps are also taking a step in local delivery space. Uber experimented with \textbf{UberFresh} in the past (2014). Now it has successfully launched \textbf{UberEats}, which serves worldwide.  It aims to deliver fast food from local restaurant chains in about 10 minutes or less. We already know Zomato, Swiggy and Foodpanda are working in this area, but taxi services entering in this area is totally a new thing. Ola has acquired foodpanda and aims to do the same things as UberEats.
Gett is another taxi app which brings items from shops and restaurants , it has joined hands with WunWun , a delivery app to facilitate the same. It delivers every product you can buy from a local shop. 

\par Taxi companies are competing with big players like amazon and google . Although google is new in this space , being a tech giant , it has plenty of resources. Apart from that they have are giving major competition to delivery startups like Postmates , DoorDash and Instacart. We will talk about these startups in brief.

An important point to note is, these type of local delivery services are currently not available in India. However, food delivery services like zomato , swiggy and foodpanda are quite popular.
Now coming to services which have quite a resemblance with our application.
Local delivery startups like Instacart, DoorDash and  Postmates are getting quite popular in the US.

Postmates works with both stores and restaurants to provide services to its customers. Currently it serves 14 cities across the US. An order placed by a user ,  is purchased and delivered by a local courier which they call “hipster on wheels” . It closely resembles what we have implemented in our android application. 

Instacart, it focuses more on grocery shopping. It sends a local shopper to a grocery store you wanted to shop from, goods are delivered within an hour.

DoorDash helps in doing local delivery from restaurants . It takes about 45 minutes and serves areas in Silicon Valley , Los Angeles , Boston and Chicago.

Our application is quite simple and works on more or less on the same concepts described in this section. However , there are some key differences in our design and working style of application.

\begin{comment}
\begin{enumerate}[label=\alph*).]

\end{enumerate}
\end{comment}

%------------------------------------------------------------------------------------------------------------------

\chapter{Software Requirement Specification}
\section{Introduction}
\subsection{Purpose}
The purpose of this report is to describe and examine all assorted ideas that have come up to define our application and to give a detailed description of software requirements and technologies used to develop our application “Social Delivery”. It also describes its user interface and specifies all its functional and non-functional requirements as well as design constraints. Various use-cases are discussed and Use-case diagrams are also used to explain the structure of the application and provide all actions performed by various actors.

\subsection{Scope}
Our “\textbf{Social Delivery}” application will help people around the world who are using our app as Shoppers to order market products from their homes and get them delivered at their doorstep for a small delivery charge. Deliveries will be made by other users who are using the app as Deliverers.
The users using our app as shoppers will benefit as they are getting their delivery to wherever they want without going anywhere themselves .The users who are using our app as deliverers will also benefit as they will get paid for the delivery in the form of delivery charge.
In this way both the parties involved are happy.

\subsection{Overview}
The next section, \textbf{Overall Description}, of this document gives an overview of the functionalities of our application. It describes the informal requirements and is used to establish a context for the technical requirements specification in the next section. 

The third section, \textbf{Requirements}, of this document is written primarily for the developers and describes in technical terms the details of the functionality of the product. 
Both sections of the document describe the same software product in its entirety, but are intended for different audiences and thus use different language.


\section{Overall Description}
\subsection{Product perspective}

\begin{enumerate}[label=\alph*)]
\item This app is an Android based application designed on \textbf{Android Studio} platform.
\item It uses \textbf{Google Firebase} Database which is a NoSQL Real-time cloud-hosted Database. We store the profiles of all users and all the data of the orders placed by them in a JSON tree. 
\item It uses \textbf{Google Places API} to set the delivery location of an order and \textbf{Google maps} to help the Deliverers to reach the delivery location by showing them the path. 
\item Our app also uses \textbf{GPS Location} to find the current city of the user. 
\item Notification alerts are also sent to users on every status change of their orders using the \textbf{One Signal API}. The One Signal API  is an open source service which uses \textbf{Google Cloud Messaging(GCM)} in its backend.
\end{enumerate}

\subsection{Product features}

For Users using our app as Shopper :
\begin{enumerate}[label=\alph*)]
\item Place a new order.
\item Edit existing pending orders.
\item Filter the orders according to status like All, Active, Pending, Finished, Cancelled, Expired.
\item See his/her PayTM wallet balance.
\item Check the details of the deliverer who have accepted his/her order.
\item Switch to deliverer view.
\item Sign-out from the app.
\end{enumerate}

For Users using our app as Deliverer:
\begin{enumerate}[label=\alph*)]
\item Accept a pending order out of a list of all pending orders from his city location.
\item Reject an accepted order.
\item See the path to reach the delivery location for an order.
\item See his/her current location(city).
\item Filter the orders according to status like All, Active, Pending, Finished, Cancelled, Expired.
\item See his/her PayTM wallet balance.
\item Complete the delivery of an accepted order by OTP verification. 
\item Switch to shopper view.
\item Sign-out from the app.
\end{enumerate}

\subsection{User characteristics}
\begin{itemize}
\item The user must have a smartphone with Android OS and must know how to operate basic android apps.
\end{itemize}

For Users using our app as Shopper:
\begin{itemize}
\item The user must know how to use pick places from a google maps.
\item The description of the orders should be as clear as possible.
\item The estimated price range of order should be given as close as possible to the real price.
\item Appropriate expiry Date \& Time of the order should be given (if any).
\item Delivery location should be set appropriately. 
\item The user must know how to pay using PayTM wallet.
\end{itemize}

For Users using our app as Deliverer:
\begin{itemize}
\item The user should know how to use Google maps app.
\item The user must know how basic OTP verification process works.
\item Users using our app as Deliverers should choose wisely what orders they can deliver.
\end{itemize}

\subsection{Constraints}
\begin{itemize}
\item Active internet connection is required while using this app.
\item E-mail ID of the users should be valid.
\item GPS should be working properly.
\item Google maps application(optional) should be installed in the Android device to be able to use SHOW PATH functionality.
\item Database constraints:
\begin{itemize}
\item As we are using the free version of the Firebase, we are limited to a maximum of 100 active connections (users) simultaneously at a time.
\item Currently we can store a total of 1 GB data on our database.
\end{itemize}
\end{itemize}

\subsection{Assumptions and dependencies}

\begin{itemize}
\item The device which is used for our app should have Android OS version 6 or above. 
\item We are assuming that the PayTM payment service which we are using for payment of order will be available all the time and will work without any errors.
\item We are assuming the GPS location will always give correct results.
\item No more than one user who is using our app as Deliverer should Accept a single order at the same time.
\end{itemize}

\section{Requirements}

\subsection{Functional Requirements}

\textbf{Module-1 :} User Login using E-mail \& Password \\
AIM : User can login to our app using unique E-mail \& Password\\
                    INPUT : E-mail and password.\\    
                    OUTPUT :  Displays a message if username and password does not match otherwise enter into the Usage mode screen.\\
                    PROCESS : Match username and password from database.\\
\newline
\textbf{Module-2 :} User Login using Google Sign-in\\
                    AIM : User can login to our app using G-Mail ID.\\
                    INPUT : G-Mail ID (E-mail and password).\\
                    OUTPUT : enter into the Usage mode screen.\\
                    PROCESS : Use GoogleApiClient.\\
\newline
\textbf{Module-3 :} Reset Password\\
                    AIM : User can reset password using registered E-mail.\\					INPUT : Registered E-mail.\\
                    OUTPUT : Password reset link sent to registered E-mail.\\
                    PROCESS : Use Firebase Authentication functions.\\
\newline
\textbf{Module-4 :} Registration\\
					AIM : A new user can register to our app by filling the form.\\				INPUT : Name, Mobile No., Alternate Mobile No.(optional), E-mail and password.\\
					OUTPUT : E-mail verification link sent to given E-mail if E-mail valid otherwise asks to enter valid E-mail address.\\
					PROCESS : Use Firebase Authentication functions.\\
\newline
\textbf{Module-5 :} Verify E-mail\\
					AIM : User will follow the instructions given in link for E-mail verification to verify E-mail before using our app.\\
					INPUT : Press Refresh Button\\
                    OUTPUT : Enter into the Usage mode screen if E-mail verification successful else remain in E-mail verification screen.\\
                    PROCESS : Save all data in database.\\
\newline
\textbf{Module-6 :} Re-send Verification E-mail\\
					AIM : Re-send the E-mail verification link to entered E-mail while registering.\\
					INPUT : Touch Re-send Verification Mail button.\\
					OUTPUT : Verification E-mail sent to E-mail.\\
					PROCESS : Use Firebase Authentication functions.\\
\newline
\textbf{Module-7 :} Place a new order\\
					AIM : User can place a new order when using our app as Shopper by filling a form.\\
INPUT : Description or order, Category, Estimated price range minimum and maximum values, Expiry Date and Time of order(optional) and delivery location.\\
					OUTPUT : PayTM payment gateway if all details are valid else displays appropriate message.\\
					PROCESS : Check maximum and minimum value and whether all mandatory fields are filled.\\
\newline
\textbf{Module-8 :} Order payment\\
					AIM : Finish payment of the order by paying from PayTM wallet using PayTM payments SDK.\\
					INPUT : PayTM login details.\\
					OUTPUT : Order details added in Database and details updated in Database if payment successful else no change in database.\\
					PROCESS : Save all data in database and send notification to shopper.\\
\newline
\textbf{Module-9 :} Edit pending order\\
					AIM : User can edit an existing pending order when using our app as Shopper by filling a form.\\
					INPUT : Description of order, Category, Estimated price range minimum value, Expiry Date and Time of order(optional) and delivery location.\\
					OUTPUT : opens order view screen if all details are valid else displays appropriate message.\\
                    PROCESS : Save all data in database.\\
\newline
\textbf{Module-10 :} Apply a filter\\
					AIM : Displays the orders according to selected filters.\\
					INPUT : Touch the filter option that you want to apply in the Navigation drawer of Shopper or Deliverer view.\\
					OUTPUT : Recycler view containing cards of orders of chosen status only.\\
					PROCESS : Database query to fetch orders from database of chosen status.\\
\newline
\textbf{Module-11 :} Sign-out\\
					AIM : To Sign-out of the app.
					\\INPUT : Touch Sign Out button in the Navigation drawer of Shopper or Deliverer view.\\
					OUTPUT : Signs out of the app and displays the login screen.\\
					PROCESS : Use Firebase Authentication functions.\\
\newline
\textbf{Module-12 :} View as Deliverer\\
					AIM : Switch to Deliverer mode from the Shopper mode.\\
					INPUT : Touch View as Deliverer button in the Navigation drawer of Shopper view.\\
					OUTPUT : Opens the Deliverer view.\\
					PROCESS : Call the intent for the Deliverer view.\\
\newline
\textbf{Module-13 :} View as Shopper\\
					AIM : Switch to Shopper mode from the Deliverer mode.\\
					INPUT : Touch View as Shopper button in the Navigation drawer of Deliverer view.\\
					OUTPUT : Opens the Shopper view.\\
					PROCESS : Call the intent for the Shopper view.\\
\newline
\textbf{Module-14 :} Show Path\\
					AIM : Shows the path to delivery location of an order to the Deliverer using Google maps.\\
					INPUT : touch Show Path button.\\
					OUTPUT : Opens the path in Google maps.\\
					PROCESS : Calls the Google maps app.\\
\newline
\textbf{Module-15 :} Accept an order\\
					AIM : Accept a pending order.\\
					INPUT : touch Accept button in Deliverer order detailed view.\\
					OUTPUT : Enables Complete Order button.\\
					PROCESS : Update details in database  and send notification to shopper.\\
\newline
\textbf{Module-16 :} Reject an order\\
					AIM : Reject an accepted order.\\
					INPUT : touch Reject button in Deliverer order detailed view.\\
                    OUTPUT : Disables Complete Order button.\\
                    PROCESS : Update details in database and send notification to shopper.\\
\newline
\textbf{Module-17 :} Complete an order\\
					AIM : Complete the delivery of an accepted order.\\
					INPUT : touch Complete Order button in Deliverer order detailed view.\\
					OUTPUT : opens SENT OTP screen.\\
					PROCESS : calls intent for SENT OTP screen.\\
\newline
\textbf{Module-18 :} Send OTP\\
					AIM : Send OTP to the shopper from the deliverer.\\
					INPUT : actual price of item bought.\\
					OUTPUT : opens ENTER OTP screen.\\
                    PROCESS : Update details in database and send notification to shopper.\\
\newline
\textbf{Module-18 :} Finish Order\\
                    AIM : Finish the delivery of an order.
INPUT : OTP as received from shopper.\\
                    OUTPUT : Delivery finished if correct OTP entered otherwise displays message.\\
                    PROCESS : Update details in database and send notification to shopper.\\


\subsection{Non- functional Requirements}

\subsubsection{Performance Requirements}
The response time should be less and the actions taken by user should not take much time. The access to database should be fast enough. The amount of graphics/animations should be such that it ensure all the above points.

\subsubsection{Availability}
The application will run 24x7 if internet connection is available.

\subsubsection{Security}
The Database used to store all the application related data should be secure against any malicious activity.

\subsubsection{Reliability}
This application has been tested on Android Oreo, Android Marshmallow, custom UIs like MIUI (Xiaomi phones), EMUI (Honor phones) and will run on any android phone having Android SDK version 6 or above.

\subsubsection{Fault Tolerance}
The data stored in the database should not get corrupted by any means like system crash, power failure or lost internet connectivity. The information stored in database should be resistance to any unwanted changes.

\subsubsection{Portability}
The application should work on any android device like mobile phones, tablets, fablets etc.

\subsubsection{Maintainability}
It should be easy to add any new feature as per requirement anytime.

\subsection{External Interface Requirements}
\subsubsection{User Interfaces}

\begin{itemize}
\renewcommand{\labelitemi}{$\rightarrow$}
\item For Front-end : XML(eXtensible Markup Language) and Android(Java)
\item For Back-end : Google Firebase Database            
\end{itemize}

\subsubsection{Hardware Interfaces}
Any device with Android OS like mobile phone, tablet or fablet.

\subsubsection{Software Interfaces}
\begin{itemize}[label=$\diamond$]
\item Softwares :
\begin{itemize}[label=$\rightarrow$]
\item Android Studio : Development environment for Google's Android operating system.
\end{itemize}
\item Languages :
\begin{itemize}[label=$\rightarrow$]
\item For Front-end : XML(eXtensible Markup Language) and Android(Java)
\item For Back-end : Java
\end{itemize}

\item Tools / Libraries:
\begin{itemize}[label=$\rightarrow$]
\item Google Sign-in
\item E-mail verification (using Firebase Authentication)
\item Google Places API
\item Google Maps
\item Location
\item PayTM payments SDK
\item OneSignal API (for Notifications)
\item Lucidchart.com (for diagrams)
\end{itemize}

\item Databases :
\begin{itemize}[label=$\rightarrow$]
\item Google Firebase (NoSQL Real-time cloud-hosted Database) :
\begin{enumerate}
\item Data is synced across all clients in realtime, and remains available when your app goes offline.
\item Data is stored as JSON (JavaScript Object Notation) tree.
\end{enumerate}
\end{itemize}

\end{itemize}

\subsubsection{Communications Interfaces}
Internet connection in the device through any medium like 3G, 4G or Wifi.

\subsection{Use-cases}
This application is mainly used for shopping for items from local stores.
\begin{itemize}
\item In our current design of application we have a community of shoppers and deliverers using our app.
\item Any user can choose to become a shopper or deliverer . Some days you can be a shopper and on other days you can be a deliverer.
\end{itemize}

\textbf{Use case 1 :  as a shopper.}\\
You want to buy some item from a shop but you are too busy to go there . You can simply use this application and someone will deliver the product at your doorstep. \\

\textbf{Use case 2 : as a shopper with some urgency.}\\
You want  some item before a certain time . Just put your deadline as expiry date . It is useful when you don’t want anyone to accept your order  after your deadline . 
It is also obvious that deliverers will process the order which has more incentive. We can have provision for tips also . Although , it is not included in current version of app.\\

\textbf{Use case 3: as a deliverer who is in market.}\\
You are currently in main market of your city and you are done with your shopping . You open our app and find many users want goods which are easily available in this market  . It’s an earning opportunity for you.\\

\textbf{Use case 4: as a deliverer with a will to earn.}\\
Today you have some free time .  You open the application and see many orders with a familiar destination and marketplace. You decide  it is a good opportunity to earn something . You accept some orders and process them to earn incentives (delivery charge).\\
The above four use cases clearly describe the utility of our application . However , with the same design and some changes we can easily convert this application to have more use cases.\\

Example of few such use cases :-
\begin{enumerate}[label=\alph*.]
\item Partnering with local shops and having centralised system of delivery between the shop and its customers. 
\item Having listing of shops and using the application to buy goods from shops.
\item Partnering with restaurants for booking tables or ordering food items from their menu.
\end{enumerate}
%---------------------------------------------

\chapter{Application Design}
\section{Software Architecture}

\section{UI Design}
We have made an android application which helps people shop from local stores. Using our application you can order any goods or food items and it will be delivered to you by one of our deliverer.\\

As our app is a kind of a transportation app, we initially thought of making the user interface similar to that of the \textbf{OLA} app.\\
The operations would have been like follows:\\

For placing a new order:
\begin{itemize}
\item A map would have been there which will show all the deliverers, out of which a deliverer would be alloted to us for our delivery.
\item the selected deliverer would be notified of our order and that deliverer would buy our order and deliver it to us at the location chosen at the time of placing the order.
\end{itemize}

But that would type of interface would have brought a lot of problems like:
\begin{itemize}
\item As the deliverer is selected randomly and not of his free will, there are high chances of order rejection, as that deliverer may not want to go to the place from where the order is to be purchased or does not intend to go to the delivery location.
\item The deliverer may not want to deliver your order because of low delivery charge (delivery incentive).
\item So, we may have to get our order rejected a lot of times before we could actually get a deliverer who is willing to deliver our order to us.
\end{itemize}

So, we needed to ensure that the deliverer \textbf{chooses} to deliverer our order.\\
\newline
We solved these problems by choosing a different User interface:\\
The new operations are like follows:\\
For placing a new order:
\begin{itemize}
\item Instead of using a map and choosing a deliverer at the time of placing the order, we created a list of all pending orders, just like unread mails in \textbf{google mail}. This method also reduces our dependency on google apis which are costly.
\item We also have filters for pending , active , cancelled , finished and expired orders which can be compared with different kind of mail filters we have in gmail app like inbox , outbox , drafts , social etc.
\item So now every newly placed order will be placed in the list of pending orders which is a global list visible to all the users using our app as a deliverer.
\item All deliverers can check the details of your order, and the deliverers who are interested in delivering your order will \textbf{Accept} your order and will deliver it to you.
\item You will be notified of  as soon as someone \textbf{Accepts} your order.
\end{itemize}
This list of pending orders is implemented using \textbf{Recycler View of cards} which is a very efficient method for displaying a big list of objects in any android application.

\subsection{Login Screen}
\begin{itemize}
\item This is the first screen which is displayed when the app is opened.
\item There are two methods for login:
\begin{itemize}[label=$\rightarrow$]
\item E-mail \& Password
\item Google Sign-in
\end{itemize}

\end{itemize}

%---------------------------------------------
\chapter{Business Model}
Our application can be used as a Business application or it can be used among friends for their logistic and shopping  needs.

As a business plan we are charging shoppers some delivery charge for the service we provide. The deliverer gets some incentive for delivering the order.

The delivery charge can be shared between us and the delivery person. When we find less deliverers are willing to use our app, we may increase their incentive to make up for the number of deliverers. 

Also, we initially planned to launch this app in our college. We didn\textquotesingle t do so due to time constraints and some other reasons.

For more smooth functioning of our application we may collaborate with third party delivery services.

We will discuss more business opportunities in future work section of our report.


\chapter{Current status of Project}
Currently all the basic features mentioned in this report are working properly.\\

We have tested our application for various scenarios like
\begin{enumerate}[label=\roman*)]
\item  if network is not working properly we are not allowing any action such as accept or reject as it may lead to confusion between users.
\item we are asking for location so that one can only accept same city orders.
\item  Updating the user screen as soon as status of any of the order changes to avoid ambiguity that may arise if a user cancels an active order.
\item Since our app is quite realtime (order status might change anytime) , we have taken care of any of the corner cases that may cause conflicts in status of orders. However , it may happen that some cases did not cross our mind, you may suggest if you find any of those. We will fix issues as soon as we find them.

\par We don\textquotesingle t have access to all the features that PayTm gateway provides . Since we have testing credentials for the gateway , we can only simulate the process of making of payment currently  . However, the gateway is fully functional .
To get a genuine merchant id , we need to register a company name.

\end{enumerate}


\chapter{Future scope}
While discussing Use case section of Software requirement Specification (SRS) of our project , we told about some use cases which can be thought of .
We may modify our application to suit specific needs.\\

Example of few such use cases :-
\begin{enumerate}[label=\Roman*]
\item Partnering with local shops and having centralised system of delivery between the shop and its customers. 
\item Having listing of shops and using the application to buy goods from shops.
\item Partnering with restaurants for booking tables or ordering food items from their menu.

\end{enumerate}


%---------------------------------------------------------------------------------------------------

%---------------------------------------------------------------------------------------------------
\begin{comment}
\chapter{Future Work}
\begin{enumerate}[label=\arabic*)]
\item Cleaning of datasets and preparing it for training and testing of our model.
\item Implementing proposed algorithm(SFA) and building our model with help of it.
\item Calculation of accuracy and selection of appropriate hyper parameters for optimal results.
\item Comparing our results with results of conventional methods used for cross domain sentimental analysis(e.g. Use of CNN).
\end{enumerate}
\end{comment}


\bibliographystyle{IEEEtran}
\bibliography{references}
\end{document}